%%%%%%%%%%%%%%%%%%%%%%%%%%%%%%%%%%%%%%%%%
% "ModernCV" CV and Cover Letter
% LaTeX Template
% Version 1.2 (25/3/16)
%
% This template has been downloaded from:
% http://www.LaTeXTemplates.com
%
% Original author:
% Xavier Danaux (xdanaux@gmail.com) with modifications by:
% Vel (vel@latextemplates.com)
%
% License:
% CC BY-NC-SA 3.0 (http://creativecommons.org/licenses/by-nc-sa/3.0/)
%
% Important note:
% This template requires the moderncv.cls and .sty files to be in the same 
% directory as this .tex file. These files provide the resume style and themes 
% used for structuring the document.
%
%%%%%%%%%%%%%%%%%%%%%%%%%%%%%%%%%%%%%%%%%

%----------------------------------------------------------------------------------------
%	PACKAGES AND OTHER DOCUMENT CONFIGURATIONS
%----------------------------------------------------------------------------------------

\documentclass[11pt,letterpaper,sans]{moderncv} % Font sizes: 10, 11, or 12; paper sizes: a4paper, letterpaper, a5paper, legalpaper, executivepaper or landscape; font families: sans or roman

\moderncvstyle{banking} % CV theme - options include: 'casual' (default), 'classic', 'oldstyle' and 'banking'
\moderncvcolor{blue} % CV color - options include: 'blue' (default), 'orange', 'green', 'red', 'purple', 'grey' and 'black'

\usepackage[scale=0.75]{geometry} % Reduce document margins
%\setlength{\hintscolumnwidth}{3cm} % Uncomment to change the width of the dates column
%\setlength{\makecvtitlenamewidth}{10cm} % For the 'classic' style, uncomment to adjust the width of the space allocated to your name

\def\prd{{Phys. Rev.} D}
\def\PRL{{Phys.Rev.} Lett}
\def\apjl{{Astrophys. J.} Lett}
\def\apj{{Astrophys. J.}}
\def\CQG{{Class. Quantum Grav.}}
\def\aaps{{A\&AS}}
\def\pasj{{PASJ}}
\def\mnras{{MNRAS}} 
\def\aapr{{A\&ARv}}
\def\aap{{A\&A}}
\def\na{{New Astronomy}}
\def\ptp{{Progress of Theoretical Physics}}
\def\apjs{{ApJS}}
\def\araa{{ARA\&A}}
\def\ssr{{Space Sci. Rev.}} 

\usepackage{etoolbox}% http://ctan.org/pkg/etoolbox
\makeatletter
\newcommand*{\emailA}[1]{\def\@emailA{#1}}
\newcommand*{\emailB}[1]{\def\@emailB{#1}}
\patchcmd{\maketitle}% <cmd>
  {\ifthenelse{\isundefined{\@email}}{}{\addtomaketitle{\emailsymbol\emaillink{\@email}}}}% <search>
  {\ifthenelse{\isundefined{\@emailA}}{}{\addtomaketitle{\emailsymbol\emaillink{\@emailA}}}%
   \ifthenelse{\isundefined{\@emailB}}{}{\addtomaketitle{\emailsymbol\emaillink{\@emailB}}}}% <replace>
  {}{}% <success><failure>
\makeatother
 
%----------------------------------------------------------------------------------------
%   NAME AND CONTACT INFORMATION SECTION
%----------------------------------------------------------------------------------------

\firstname{Justin} % Your first name
\familyname{Ellis} % Your last name

% All information in this block is optional, comment out any lines you don't need
\title{Curriculum Vitae}
\address{Jet Propulsion Laboratory, 4800 Oak Grove Drive}{Pasadena, CA 91109}
\phone[mobile]{+1 (304) 995-6951}
\emailA{Justin.A.Ellis@jpl.nasa.gov}
\emailB{justin.ellis18@gmail.com}
%\homepage{}
\social[github]{jellis18}
\social[linkedin][www.linkedin.com/in/justin-ellis-44a32788]{justin-ellis}
%----------------------------------------------------------------------------------------

\begin{document}

\makecvtitle % Print the CV title

%----------------------------------------------------------------------------------------
%	BLURB
%----------------------------------------------------------------------------------------

\small{Einstein Postdoctoral Fellow at the Jet Propulsion Laboratory/Caltech working on several data analysis tasks for pulsar timing arrays. Passionate about data analysis/science, machine learning, and data visualization. Have leadership experience working in large and small collaborations.}

%----------------------------------------------------------------------------------------
%	EDUCATION SECTION
%----------------------------------------------------------------------------------------

\section{Education}
\vspace{6pt}

\cventry{July 2011--June 2014}{PhD (Physics)}{University of Wisconsin Milwaukee}{Milwaukee, WI}{}{}
\cventry{August 2009--June 2011}{PhD student in Physics}{West Virginia University}{Morgantown, WV}{}{}
\cventry{January 2007--June 2009}{B.S. in Physics}{West Virginia University}{Morgantown, WV}{Mathematics, Astronomy minor}{}
\cventry{August 2004--January 2006}{B.S. student in Mathematics}{Shepherd University}{Shepherdstown, WV}{}{}

\section{Doctoral Thesis}
\vspace{6pt}

\cvitem{Title}{\emph{Searching for Gravitational Waves Using Pulsar Timing Arrays}}
\cvitem{Advisor}{Dr. Xavier Siemens}
\cvitem{Description}{This thesis details several new frequentist and Bayesian detection and characterization
schemes for gravitational waves using pulsar timing arrays. These methods were applied to several real and synthetic
datasets to produce stringent constraints on gravitational waves from supermassive black hole binaries.}

%----------------------------------------------------------------------------------------
%	WORK EXPERIENCE SECTION
%----------------------------------------------------------------------------------------

\section{Professional Experience}
\vspace{6pt}

\subsection{Research Experience}
\vspace{6pt}

\cventry{September 2014--Present}{Einstein Postdoctoral Fellow}{NASA Jet Propulsion Laboratory}{Pasadena, CA}{}{}
\cventry{September 2014--Present}{Visiting scholar (TAPIR group)}{California Institute of Technology}{Pasadena, CA}{}{}
\cventry{July 2011--June 2014}{Graduate Research Assistant}{University of Wisconsin Milwaukee}{Milwaukee, WI}{Advisor: Dr. Xavier Siemens}{}
\cventry{August 2009--June 2011}{Graduate Research Assistant}{West Virginia University}{Morgantown, WV}{Advisor: Dr. Maura McLaughlin}{}
\cventry{June 2008--June 2009}{Undergraduate Research Assistant}{West Virginia University}{Morgantown, WV}{Advisor: Dr. Earl Scime}{}
\cventry{August 2005--January 2006}{Undergraduate Research Assistant}{Shepherd University}{Shepherdstown, WV}{Advisor: Dr. Jason Best}{}

\subsection{Teaching Experience}
\vspace{6pt}

\cventry{March 2016}{Co-organizer of NANOGrav student workshop}{California Institute of Technology}{Pasadena, CA}{}{}
\cventry{September 2015}{Co-organizer of NANOGrav detection group workshop}{California Institute of Technology}{Pasadena, CA}{}{}
\cventry{January 2010--August 2010}{Tutor (calculus based introductory physics)}{West Virginia University}{Morgantown, WV}{}{}
\cventry{January 2008--August 2010}{Lab Instructor (calculus based introductory physics)}{West Virginia University}{Morgantown, WV}{}{}
\cventry{September 2007--December 2007}{Lab Instructor (algebra based introductory physics)}{West Virginia University}{Morgantown, WV}{}{}


%----------------------------------------------------------------------------------------
%	AWARDS SECTION
%----------------------------------------------------------------------------------------

\section{Awards}
\vspace{6pt}

\cvitem{2015}{Chair of NANOGrav Gravitational Wave Detection Working Group }
\cvitem{2014}{Einstein Fellowship (JPL/Caltech)}
\cvitem{2014}{Papastamatiou Scholarship (UWM)}
\cvitem{2013}{Distinguished Dissertation Fellowship (UWM)}
\cvitem{2013}{NASA Wisconsin Space Grant Consortium Fellowship}
\cvitem{2012}{Blue Apple Award (22nd Midwest Relativity Meeting)}
\cvitem{2012}{NASA Wisconsin Space Grant Consortium Fellowship}
\cvitem{2011}{Best Graduate Student Oral Presentation (West Virginia Academy of Science)}
\cvitem{2009}{Outstanding Physics Senior Award (WVU)}
\cvitem{2008}{Reddy Scholarship for Academic Excellence (WVU)}
\cvitem{2008}{Eberly College of Arts and Sciences Award for Academic Excellence (WVU)}


%----------------------------------------------------------------------------------------
%	COMPUTER SKILLS SECTION
%----------------------------------------------------------------------------------------

\section{Computer skills}
\vspace{6pt}

\cvitem{OS}{OSX, Linux/Unix, Windows}
\cvitem{Programming}{Python, C/C++, Fortran, Matlab, Mathematica}
\cvitem{Typography}{\LaTeX, Microsoft Office, Pages, OpenOffice, Keynote}
\cvitem{Data Science}{Scikit-learn, Pandas}


%----------------------------------------------------------------------------------------
%   AFFILIATIONS SECTION
%----------------------------------------------------------------------------------------

\section{Professional Affiliations}
\vspace{6pt}

\cvitem{American Physical Society}{Member}
\cvitem{American Astronomical Society}{Member}
\cvitem{North American Nanohertz Observatory for Gravitational waves (NANOGrav)}{Full member}
\cvitem{International Pulsar Timing Array (IPTA)}{Member}

%----------------------------------------------------------------------------------------
%   PRESENTATIONS SECTION
%----------------------------------------------------------------------------------------

\section{Recent presentations}
\vspace{6pt}

\subsection{Invited Talks}

\cvitem{April 2016}{\textit{PTA searches for Gravitational Waves: Astrophysics with non-detections}, University of Wisconsin Milwaukee CGCA Seminar, Milwaukee WI}
\cvitem{April 2016}{\textit{Astrophysics with Pulsar Timing Arrays}, Northwestern University Lunch Seminar, Evanston, IL}
\cvitem{November 2014}{\textit{Gravitational Wave Science Using Pulsar Timing Arrays}, Embry-Riddle University Colloquium, Prescott, AZ}
\cvitem{October 2014}{\textit{Pulsar Timing Arrays: A Galactic Scale Gravitational Wave Detector}, Montana State Physics Department Colloquium, Bozeman, MT}
\cvitem{October 2014}{\textit{Data Analysis for Pulsar Timing Arrays}, Montana State Physics Department Seminar, Bozeman, MT}
\cvitem{June 2014}{\textit{The IPTA continuous wave search project.}, IPTA Meeting, Banff, Canada}
\cvitem{July 2013}{\textit{Searching for Gravitational Waves using Pulsar Timing Arrays'' }, Cambridge University Seminar, Cambridge, UK}
\cvitem{June 2013}{\textit{Single Source Detection and Upper Limits with IPTA data}, IPTA Meeting, Krabi Beach, Thailand}
\cvitem{June 2012}{\textit{An Overview of Single-Source Detection Algorithms}, IPTA Meeting, Kiama, Australia}
\vspace{6pt}

\subsection{Contributed Talks}

\cvitem{June 2016}{\textit{Trans-dimensional pulsar timing data analysis}, IPTA Meeting, Stellenbosch, South Africa}
\cvitem{April 2016}{\textit{Trans-dimensional signal modeling in PTA data}, APS April Meeting, Salt Lake City, UT}
\cvitem{August 2015}{\textit{Constraining Supermassive Black Hole Binary Dynamics Using Pulsar Timing Data}, IAU Meeting, Honolulu, HI}
\cvitem{April 2015}{\textit{Preliminary NANOGrav limits on the isotropic stochastic GWB from the 9-year data release}, APS April Meeting, Baltimore, MD}
\cvitem{January 2015}{\textit{Searching for GWs using pulsar timing arrays}, AAS Meeting, Seattle, WA}
\cvitem{April 2014}{\textit{NANOGrav Limits on Gravitational Waves from Individual Supermassive Black Hole Binaries in Circular Orbits}, APS April Meeting, Savannah, GA}
\cvitem{January 2014}{\textit{I get by with a little help from my friends: Enhancing PTA sensitivity to GWs using EM counterparts'}, AAS Meeting, Washington DC}
\cvitem{October 2013}{\textit{Gravitational Wave Detection with Pulsar Timing Arrays}, CaJAGWR Seminar Caltech, Pasadena, CA}
\cvitem{October 2013}{\textit{Gravitational Wave Detection with Pulsar Timing Arrays}, JPL Seminar, Pasadena, CA}
\cvitem{August 2013}{\textit{Searching for Gravitational Waves using Pulsar Timing Arrays'}, Wisconsin Space Grant Consortium Conference, Milwaukee, WI}
\cvitem{July 2013}{\textit{Continuous Gravitational Wave Search Methods and Results from PTAs}, 10th Edoardo Amldi Conference on Gravitational Waves, Warsaw, Poland}
\cvitem{January 2013}{\textit{When Will We Detect GWs?}, Physical Applications of Millisecond Pulsars, Aspen, CO}
\cvitem{January 2013}{\textit{Gravitational Wave Searches with Pulsar Timing Data}, AAS Meeting, Long Beach, CA}
\cvitem{September 2012}{\textit{Gravitational Wave Searches with Pulsar Timing Arrays}, 22nd Midwest Relativity Meeting, Chicago, IL}
\cvitem{August 2012}{\textit{Gravitational Wave Searches with Pulsar Timing Data}, IAU Meeting, Beijing, China}
\cvitem{November 2011}{\textit{Detection Methods for Continuous Gravitational Waves Using Pulsar Timing Arrays}, 23rd Midwest Relativity Meeting, Urbana, IL}
\cvitem{June 2011}{\textit{Detection of Continuous Gravitational Waves with Pulsar Timing Arrays}, IPTA Meeting, Snowshoe, WV}
\vspace{6pt}

\subsection{Posters}

\cvitem{January 2012}{\textit{Bayesian Methods for Covariance Estimation of Pulsar Timing Residuals}, AAS Meeting, Austin, TX}
\cvitem{July 2011}{\textit{Detection Methods for Continuous Gravitational Waves Using Pulsar Timing Arrays}, 9th Edoardo Amaldi Conference on Gravitational Waves, Cardiff, Wales}
\cvitem{January 2011}{\textit{Continuous Gravitational Wave Searches in Pulsar Timing Data}, AAS Meeting, Seattle, WA}
\cvitem{June 2010}{\textit{The impact of a stochastic gravitational-wave background on pulsar timing parameters'}, IPTA Meeting, Leiden, Netherlands}
\cvitem{November 2008}{\textit{A magneto-optic probe for magnetic fluctuation measurements}, 0th Annual Meeting of the Division of Plasma Physics (DPP), Dallas Texas}

%----------------------------------------------------------------------------------------
%   PUBLICATIONS
%----------------------------------------------------------------------------------------

%\nocite{*}
%\bibliographystyle{plainyr-rev}
%\bibliography{pubs}   

\nocite{*}
\begin{thebibliography}{10}
\vspace{6pt}

\bibitem{fpe+16}
E.~{Fonseca}, T.~T. {Pennucci}, \textbf{J.~A. {Ellis}}, I.~H. {Stairs}, D.~J. {Nice},
  S.~M. {Ransom}, P.~B. {Demorest}, Z.~{Arzoumanian}, K.~{Crowter}, T.~{Dolch},
  R.~D. {Ferdman}, M.~E. {Gonzalez}, G.~{Jones}, M.~L. {Jones}, M.~T. {Lam},
  L.~{Levin}, M.~A. {McLaughlin}, K.~{Stovall}, J.~K. {Swiggum}, and W.~{Zhu}.
\newblock {The NANOGrav Nine-year Data Set: Mass and Geometric Measurements of
  Binary Millisecond Pulsars}.
\newblock {\em Submitted to ApJ}, March 2016.

\bibitem{ec16}
\textbf{J.~A. {Ellis}} and N.~J. {Cornish}.
\newblock {Transdimensional Bayesian approach to pulsar timing noise analysis}.
\newblock {\em \prd}, 93(8):084048, April 2016.

\bibitem{tve+16}
S.~R. {Taylor}, M.~{Vallisneri}, \textbf{J.~A. {Ellis}}, C.~M.~F. {Mingarelli}, T.~J.~W.
  {Lazio}, and R.~{van Haasteren}.
\newblock {Are We There Yet? Time to Detection of Nanohertz Gravitational Waves
  Based on Pulsar-timing Array Limits}.
\newblock {\em \apjl}, 819:L6, March 2016.

\bibitem{abb+16}
Z.~{Arzoumanian}, A.~{Brazier}, S.~{Burke-Spolaor}, S.~J. {Chamberlin},
  S.~{Chatterjee}, B.~{Christy}, J.~M. {Cordes}, N.~J. {Cornish}, K.~{Crowter},
  P.~B. {Demorest}, X.~{Deng}, T.~{Dolch}, \textbf{J.~A. {Ellis}}, R.~D. {Ferdman},
  E.~{Fonseca}, N.~{Garver-Daniels}, M.~E. {Gonzalez}, F.~{Jenet}, G.~{Jones},
  M.~L. {Jones}, V.~M. {Kaspi}, M.~{Koop}, M.~T. {Lam}, T.~J.~W. {Lazio},
  L.~{Levin}, A.~N. {Lommen}, D.~R. {Lorimer}, J.~{Luo}, R.~S. {Lynch}, D.~R.
  {Madison}, M.~A. {McLaughlin}, S.~T. {McWilliams}, C.~M.~F. {Mingarelli},
  D.~J. {Nice}, N.~{Palliyaguru}, T.~T. {Pennucci}, S.~M. {Ransom},
  L.~{Sampson}, S.~A. {Sanidas}, A.~{Sesana}, X.~{Siemens}, J.~{Simon}, I.~H.
  {Stairs}, D.~R. {Stinebring}, K.~{Stovall}, J.~{Swiggum}, S.~R. {Taylor},
  M.~{Vallisneri}, R.~{van Haasteren}, Y.~{Wang}, W.~W. {Zhu}, and {The
  NANOGrav Collaboration}.
\newblock {The NANOGrav Nine-year Data Set: Limits on the Isotropic Stochastic
  Gravitational Wave Background}.
\newblock {\em \apj}, 821:13, April 2016.

\bibitem{zsd+15}
W.~W. {Zhu}, I.~H. {Stairs}, P.~B. {Demorest}, D.~J. {Nice}, \textbf{J.~A. {Ellis}},
  S.~M. {Ransom}, Z.~{Arzoumanian}, K.~{Crowter}, T.~{Dolch}, R.~D. {Ferdman},
  E.~{Fonseca}, M.~E. {Gonzalez}, G.~{Jones}, M.~L. {Jones}, M.~T. {Lam},
  L.~{Levin}, M.~A. {McLaughlin}, T.~{Pennucci}, K.~{Stovall}, and
  J.~{Swiggum}.
\newblock {Testing Theories of Gravitation Using 21-Year Timing of Pulsar
  Binary J1713+0747}.
\newblock {\em \apj}, 809:41, August 2015.

\bibitem{ccs+15}
S.~J. {Chamberlin}, J.~D.~E. {Creighton}, X.~{Siemens}, P.~{Demorest},
  \textbf{J.~A. {Ellis}}, L.~R. {Price}, and J.~D. {Romano}.
\newblock {Time-domain implementation of the optimal cross-correlation
  statistic for stochastic gravitational-wave background searches in pulsar
  timing data}.
\newblock {\em \prd}, 91(4):044048, February 2015.

\bibitem{abb+15b}
Z.~{Arzoumanian}, A.~{Brazier}, S.~{Burke-Spolaor}, S.~J. {Chamberlin},
  S.~{Chatterjee}, J.~M. {Cordes}, P.~B. {Demorest}, X.~{Deng}, T.~{Dolch},
  \textbf{J.~A. {Ellis}}, R.~D. {Ferdman}, N.~{Garver-Daniels}, F.~{Jenet}, G.~{Jones},
  V.~M. {Kaspi}, M.~{Koop}, M.~T. {Lam}, T.~J.~W. {Lazio}, A.~N. {Lommen},
  D.~R. {Lorimer}, J.~{Luo}, R.~S. {Lynch}, D.~R. {Madison}, M.~A.
  {McLaughlin}, S.~T. {McWilliams}, D.~J. {Nice}, N.~{Palliyaguru}, T.~T.
  {Pennucci}, S.~M. {Ransom}, A.~{Sesana}, X.~{Siemens}, I.~H. {Stairs}, D.~R.
  {Stinebring}, K.~{Stovall}, J.~{Swiggum}, M.~{Vallisneri}, R.~{van
  Haasteren}, Y.~{Wang}, W.~W. {Zhu}, and {NANOGrav Collaboration}.
\newblock {The NANOGrav Nine-year Data Set: Observations, Arrival Time
  Measurements, and Analysis of 37 Millisecond Pulsars}.
\newblock {\em submitted to ApJ}, 2015.


\bibitem{abb+14}
Z.~{Arzoumanian}, A.~{Brazier}, S.~{Burke-Spolaor}, S.~J. {Chamberlin},
  S.~{Chatterjee}, J.~M. {Cordes}, P.~B. {Demorest}, X.~{Deng}, T.~{Dolch},
  \textbf{J.~A. {Ellis}}, R.~D. {Ferdman}, N.~{Garver-Daniels}, F.~{Jenet}, G.~{Jones},
  V.~M. {Kaspi}, M.~{Koop}, M.~T. {Lam}, T.~J.~W. {Lazio}, A.~N. {Lommen},
  D.~R. {Lorimer}, J.~{Luo}, R.~S. {Lynch}, D.~R. {Madison}, M.~A.
  {McLaughlin}, S.~T. {McWilliams}, D.~J. {Nice}, N.~{Palliyaguru}, T.~T.
  {Pennucci}, S.~M. {Ransom}, A.~{Sesana}, X.~{Siemens}, I.~H. {Stairs}, D.~R.
  {Stinebring}, K.~{Stovall}, J.~{Swiggum}, M.~{Vallisneri}, R.~{van
  Haasteren}, Y.~{Wang}, W.~W. {Zhu}, and {NANOGrav Collaboration}.
\newblock {Gravitational Waves from Individual Supermassive Black Hole Binaries
  in Circular Orbits: Limits from the North American Nanohertz Observatory for
  Gravitational Waves}.
\newblock {\em \apj}, 794:141, October 2014.


\bibitem{sejr13}
X.~{Siemens}, \textbf{J.~A. {Ellis}}, F.~{Jenet}, and J.~D. {Romano}.
\newblock The stochastic background: scaling laws and time to detection for
  pulsar timing arrays.
\newblock {\em Classical and Quantum Gravity}, 30(22):224015, 2013.

\bibitem{esvh13}
\textbf{J.~A. {Ellis}}, X.~{Siemens}, and R.~{van Haasteren}.
\newblock {An Efficient Approximation to the Likelihood for Gravitational Wave
  Stochastic Background Detection Using Pulsar Timing Data}.
\newblock 769:63, May 2013.

\bibitem{e13}
\textbf{J.~A. {Ellis}}.
\newblock A bayesian analysis pipeline for continuous gw sources in the pta
  band.
\newblock {\em Classical and Quantum Gravity}, 30(22):224004, 2013.

\bibitem{dfg+13}
P.~B. {Demorest}, R.~D. {Ferdman}, M.~E. {Gonzalez}, D.~{Nice}, S.~{Ransom},
  I.~H. {Stairs}, Z.~{Arzoumanian}, A.~{Brazier}, S.~{Burke-Spolaor}, S.~J.
  {Chamberlin}, J.~M. {Cordes}, \textbf{J.~A. {Ellis}}, L.~S. {Finn}, P.~{Freire},
  S.~{Giampanis}, F.~{Jenet}, V.~M. {Kaspi}, J.~{Lazio}, A.~N. {Lommen},
  M.~{McLaughlin}, N.~{Palliyaguru}, D.~{Perrodin}, R.~M. {Shannon},
  X.~{Siemens}, D.~{Stinebring}, J.~{Swiggum}, and W.~W. {Zhu}.
\newblock {Limits on the Stochastic Gravitational Wave Background from the
  North American Nanohertz Observatory for Gravitational Waves}.
\newblock 762:94, January 2013.

\bibitem{esc12}
\textbf{J.~A. {Ellis}}, X.~{Siemens}, and J.~D.~E. {Creighton}.
\newblock {Optimal Strategies for Continuous Gravitational Wave Detection in
  Pulsar Timing Arrays}.
\newblock 756:175, September 2012.

\bibitem{ejm12}
\textbf{J.~A. {Ellis}}, F.~A. {Jenet}, and M.~A. {McLaughlin}.
\newblock {Practical Methods for Continuous Gravitational Wave Detection Using
  Pulsar Timing Data}.
\newblock 753:96, July 2012.

\bibitem{esc12b}
\textbf{J.~A. {Ellis}}, X.~{Siemens}, and S.~{Chamberlin}.
\newblock {Results of the First IPTA Closed Mock Data Challenge}.
\newblock {\em arXiv:1210:5274}, October 2012.

\bibitem{emv11}
\textbf{J.~A. {Ellis}}, M.~A. {McLaughlin}, and J.~P.~W. {Verbiest}.
\newblock {The impact of a stochastic gravitational-wave background on pulsar
  timing parameters}.
\newblock 417:2318--2329, November 2011.

\bibitem{pet+09}
W.S. Przybysz, \textbf{J.~A. {Ellis}}, S.C. Thakur, A.~Hansen, R.A. Hardin, S.~Sears, and
  E.E. Scime.
\newblock A magneto-optic probe for magnetic fluctuation measurements.
\newblock {\em Rev Sci Instrum}, 80(10):103502, 2009.

\end{thebibliography}











\end{document}